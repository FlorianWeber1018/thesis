@misc{tcpDef,
author = {Postel, J},
publisher = {USC/Information Sciences Institute},
title = {RFC: 793, Internet Protocol - DARPA Internet Program Protocol Specification},
month = {September},
year = {1981},
url = {https://tools.ietf.org/html/rfc793},
}
@Inbook{Schubert:2007:2,
author="Matthias Schubert",
title="Hier spricht der Theoretiker: Relational und andere Datenbanksysteme",
bookTitle="Datenbanken: Theorie, Entwurf und Programmierung relationaler Datenbanken",
year="2007",
publisher="Teubner",
address="Wiesbaden",
pages="31--39",
chapter="2",
abstract="Wenn wir nach einem theoretischen Konzept f{\"u}r eine elektronische Speicherung und Verwaltung suchen, kommen wir sehr bald auf das Konzept einer Tabelle. Damit sind wir schon unterwegs zur Welt der relationalen Datenbanken, die das Hauptthema dieses Buches bilden. Aber: Freuen Sie sich nicht zu fr{\"u}h: Wir werden in diesem Kapitel keineswegs kl{\"a}ren k{\"o}nnen, worin das theoretische Konzept der relationalen Datenbanken besteht. Wir machen lediglich einen bescheidenen Anfang. Und am Anfang eines relationalen Datenbanksystems steht die Tabelle - sie steht dort so fest und eindrucksvoll, dass viele Benutzer relationaler Systeme Ihnen nicht erkl{\"a}ren k{\"o}nnen, warum relationale Datenbanken nicht viel besser tabel larische Datenbanken hei$\beta$en sollten.",
isbn="978-3-8351-9108-2",
doi="10.1007/978-3-8351-9108-2_2",
url="https://doi.org/10.1007/978-3-8351-9108-2\_2"
}
@Inbook{Schubert:2007:3,
author="Matthias Schubert",
title="„Das wird teuer`` --- der EDV-Spezialist tritt auf",
bookTitle="Datenbanken: Theorie, Entwurf und Programmierung relationaler Datenbanken",
year="2007",
publisher="Teubner",
address="Wiesbaden",
pages="40--49",
chapter="3",
abstract="Der Anwender hat bereits gesprochen und Forderungen f{\"u}r ein elektronisches Datenspeiche rungssystem aufgestellt und der Theoretiker hat angefangen, nachzudenken und Konzepte daf{\"u}r zu entwickeln. Da fehlt nat{\"u}rlich noch jemand: der EDV-Spezialist. Es wird Zeit, zu h{\"o}ren, was er zu sagen hat. Sein Standpunkt wird sein:1.Wir brauchen Software --- und zwar jede Menge.2.Wir brauchen Man Power f{\"u}r die Betreuung --- die Administration --- der Datenbank.",
isbn="978-3-8351-9108-2",
doi="10.1007/978-3-8351-9108-2_3",
url="https://doi.org/10.1007/978-3-8351-9108-2\_3"
}
@Inbook{Studer:2016:2,
author="Thomas Studer",
title="Das Relationenmodell",
bookTitle="Relationale Datenbanken: Von den theoretischen Grundlagen zu Anwendungen mit PostgreSQL",
year="2016",
publisher="Springer Berlin Heidelberg",
address="Berlin, Heidelberg",
pages="9--21",
chapter="2",
abstract="Die Grundidee des Relationenmodells ist es, Daten in Form von Relationen abzuspeichern. In diesem Kapitel f{\"u}hren wir die wesentlichen Begriffe des Relationenmodells ein, wie beispielsweise Attribut, Dom{\"a}ne, Schema und Instanz. Ausserdem diskutieren wir das Konzept des Prim{\"a}rschl{\"u}ssels eines Schemas. Im letzten Abschnitt behandeln wir Integrit{\"a}tsbedingungen auf einem Datenbankschema. Insbesondere definieren wir die Bedeutung von Fremdschl{\"u}sseln, unique Constraints und not null Constraints.",
isbn="978-3-662-46571-4",
doi="10.1007/978-3-662-46571-4_2",
url="https://doi.org/10.1007/978-3-662-46571-4\_2"
}
@Inbook{Scholz:2005,
author = {Scholz, Peter},
title="Echtzeit, Echtzeitsysteme, Echtzeitbetriebssysteme",
bookTitle="Softwareentwicklung eingebetteter Systeme: Grundlagen, Modellierung, Qualit{\"a}tssicherung",
year="2005",
publisher="Springer Berlin Heidelberg",
address="Berlin, Heidelberg",
pages="39--73",
chapter="3",
abstract="Unter der Echtzeitf{\"a}higkeit eines Betriebssystems versteht man in erster Linie dessen reale F{\"a}higkeit, in einer gegebenen Betriebsumgebung alle anstehenden Aufgaben und Funktionen unter allen Betriebszust{\"a}nden immer rechtzeitig und ohne Ausnahme erledigen zu k{\"o}nnen. „Rechtzeitig`` oder „in Echtzeit`` versteht sich somit nicht als exakte wissenschaftliche Definition, sondern als sehr variable Gr{\"o}{\ss}e, die sich nach den jeweiligen (Echtzeit-) Anforderungen der spezifischen Anwendungen und deren zeitlichen Rahmenbedingungen orientiert und ausrichtet. Ein Echtzeitsystem ist also ein eingebettetes System, das Echtzeitanforderungen besitzt und dann ggf. mit Hilfe eines Echtzeitbetriebssystems implementiert werden kann.",
isbn="978-3-540-27522-0",
doi="10.1007/3-540-27522-3_3",
url="https://doi.org/10.1007/3-540-27522-3\_3"
}

@article{McAfee.2006,
 author = {Luber, Stefan},
 year = {2018},
 title = {Definition: Was ist OPC UA?},
 journal = {MIT Sloan Management Review}
}
@misc{opcua:2018,
  author = {Luber, Stefan},
  publisher = "BigData-Insider.de",
  title = "Definition: Was ist {OPC UA}?",
  month = "M{\"a}rz",
  year = "2018",
  url = "https://www.bigdata-insider.de/was-ist-opc-ua-a-698144/",
}