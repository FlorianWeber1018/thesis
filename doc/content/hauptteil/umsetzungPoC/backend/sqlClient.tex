\subsection{SqlClient}
Die \eigenName{SqlClient} Klasse verwaltet die Verbindung zum \ac{sql} Server. \\Auf dem \ac{sql} Server sind alle Daten gespeichert, die persistiert werden müssen. 
Das Klassendiagramm der \eigenName{SqlClient} Klasse ist in \refFig{fig:backend:classDiag:SqlClient} dargestellt.
Die Klasse bedient sich dabei der offiziellen Bibliothek \eigenName{MariaDB Connector/C API} der OpenSource Datenbank-Engines MariaDB.
\begin{figure}[ht]
  \centering
  \includegraphics[width=\textwidth]{content/hauptteil/umsetzungPoC/backend/uml/classesOfOverview/SqlClient.pdf}
  \caption{Klassediagramm der Klasse \eigenName{SqlClient}}
  \label{fig:backend:classDiag:SqlClient}
\end{figure}
Die Datenstruktur des Servers ist in Abschnitt \ref{subsec:dataBackend} beschrieben und dort in \refFig{img:erd} dargestellt.
Diese Struktur ist auf dem \ac{sql} Server vorhanden und wird durch das Skript in \refList{list:createSqlTables} auf dem externe Server konstruiert.
Die \eigenName{SqlClient} Klasse kann mit der Methode \eigenName{executeScript} ein \ac{sql} Skript als Datei ausführen, indem sie es in einzelne Querys zerteilt und diese dann einzeln an den Server sendet.
Das Zerteilen eines Skript und das Vorbereiten zur Ausführung, ist in der Methode \eigenName{prepareScript} implementiert. 
Beim Testen der Methode hat sich herausgestellt, dass diese Funktionalität der \eigenName{sql} Client Klasse Skripte auszuführen, schwieriger zu implementieren ist als anfangs vermutet.
Dies liegt daran, dass das Definieren von \emph{stored Procedures} (siehe Abschnitt \ref{subsec:storedProc}) es nötigt macht das Semikolon als Trennzeichen zu ändern, 
da in einem Query der Quellcode vorhanden ist der selbst aus mehreren Querys besteht die mit einem Semikolon getrennt sind.
Ist ein Skript vorbereitet, so können die einzelnen Querys des Skripts durch den Aufruf der Methode \eigenName{sendCUD} auf dem Server ausgeführt werden.
Die Methode \eigenName{sendCUD} bedient sich dabei der Methode \eigenName{sendCommand}, die die Query an den Server weiterleitet.
Da die Verbindung zum \ac{sql} Server eine geteilte Resource ist, ist der Zugriff darauf durch \eigenName{sendCommand} durch ein Mutex geschützt, sodass immer nur ein Thread gleichzeitig ein Query senden kann.
Die Klasse \eigenName{SqlClient} bietet viele Methoden um die Daten in der Datenbank abzufragen oder zu verändern.
Um Daten abzufragen werden meistens Querys ausgeführt die mit \eigenName{Select} beginnen.
Allerdings sind ein paar komplizierte Abfragen in \emph{stored Procedures} auf dem Server ausgelagert.
Diese Prozeduren werden durch das Skript \eigenName{createProcedures} (siehe Anhang \refList{})
%kurze Erwähnung sinn der klasse
%datenaustausch zwischen SqlClient zu abgeleitetrer...(dispatcher.... ) (I/O)

%beschreibung msg klasse mit diagram
\begin{figure}[ht]
  \centering
  \includegraphics[width=\textwidth]{content/hauptteil/umsetzungPoC/backend/uml/classesOfOverview/sql_message.pdf}
  \caption{Klassediagramm der Klasse \eigenName{sql\_message}}
  \label{fig:backend:classDiag:sqlMsg}
\end{figure}
%dispatcher codeausschnitt
%beschreibung dispatcher