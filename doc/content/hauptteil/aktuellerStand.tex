\chapter{Aktueller Stand der Technik}
In diesem Kapitel werden die führenden \ac{scada} Systeme der drei größten Automatisierungstechnik Unternehmen der Welt \citep{umsatzAutomatisierungstechnik} kurz vorgestellt.
\section{Compact HMI}
Das \ac{scada} System von \emph{ABB} am Markt heist \eigenName{Compact HMI 6.0 Overview}.
Es ist lauffähig auf \emph{Windows 8.1}, \emph{Windows 10 2016 LTSB}, \emph{Windows Server 2012 R2} sowie \emph{Windows Server 2016} \citep{abbOverview}.
Dieses System bietet als Schnittstellen \ac{opc} (DA, HDA, AE), Comli, SattBus, Modbus Serial sowie Modbus TCP an \citep{abbOverview}.
\acp{plc} werden bevorzugt über die \ac{opc} Schnittstelle in das \ac{scada} System integriert \citep{abbOverview}.
Auserdem integriert das System den Versand von E-Mails oder SMS \citep{abbOverview}.
\section{WinCC V7}
Das \ac{scada} System des Herstellers \emph{Siemens} ist \eigenName{SIMATIC WinCC V7}.
Es bietet eine Vielzahl an Optionen die verschiedene Funktionen implementieren. 
Laut \citet{siemensOverview} ist es ein Sektor- und Technologieneutrales System das über sogenannte \eigenName{WINCC V7 options}
skalierbar ist. Es bietet außerdem eine Weboberfläche um über das Internet zu steuern und Werte zu überwachen \citep{siemensOverview}.
\section{OpenEnterprise}
\eigenName{OpenEnterprise} ist das aktuelle \ac{scada} System des Herstellers Emerson.
Es setzt sich aus einer Vielzahl von Teilsystemen zusammen.
Jedes dieser Teilsysteme ist ausschließlich unter Microsoft Windows lauffähig \citep{emersonOverview}.
Zur Anbindung an externe Datenbanksysteme, untestützt \eigenName{OpenEnterprise} die Verbindung über \ac{odbc} Treiber \citep{emersonOverview}.