\chapter{\acl{poc}}
Ein \ac{poc} ist eine Möglichkeit um die Realisierbarkeit des Systementwurfs aus Kapitel \ref{chapter:Systementwurf} zu belegen.
In diesem  \ac{poc} wird sich auf die wesentliche Funktionalität eines \ac{scada} Systems beschränkt.
Dabei wird anhand der Architektur in Abschnitt \ref{sec:arch} eine Webapplikation auf Basis von \emph{Vue.js} als Frontend, 
sowie eine ausführbare Anwendung als Backend in C++ programmiert und anschließend zusammen mit einem Webserver und einem \ac{sql} Server getestet.
Das Frontend ist wie in der Architektur bereits geplant implementiert. 
Die Dokumentation des \acp{poc} des Backends ist in dem folgenden Abschnitt \ref{sec:poc:backend} enthalten.
\section{Backend}\label{sec:poc:backend}
\begin{figure}[ht]
  \centering
  %\includegraphics[width=\textwidth]{content/hauptteil/systemEntwurf/res/LayoutFrontend.pdf} DA DES MINIMALIFIZIERTE DIAG REIN
  \caption[Klassediagramm Backend]{Klassediagramm}
  \label{fig:backend:classDiag}
\end{figure}
%beschreibung diagramm
%erwähnung datenaustausch zwischen basisklassen zu abgeleitetrer...(dispatcher.... )
  %endlich anzahl an fkt für daten backend -> basisklassen
  %pro basisklasse ein dispatcher im Backend (virtual) --> Verweis auf Kapitel
%erwähnung message klassen die geparsed wwerden


\subsection{WebsocketServer}
%kurze Erwähnung sinn der klasse
%datenaustausch zwischen WebsocketServer zu abgeleitetrer...(dispatcher.... ) (I/O)
%beschreibung msg klasse mit diagram
%dispatcher codeausschnitt
%beschreibung dispatcher
\subsection{SqlClient}
%kurze Erwähnung sinn der klasse
%datenaustausch zwischen SqlClient zu abgeleitetrer...(dispatcher.... ) (I/O)
%beschreibung msg klasse mit diagram
%dispatcher codeausschnitt
%beschreibung dispatcher
\subsection{OpcuaServer}
%kurze Erwähnung sinn der klasse
%datenaustausch zwischen OpcuaServer zu abgeleitetrer...(dispatcher.... ) (I/O)
%beschreibung msg klasse mit diagram
%dispatcher codeausschnitt
%beschreibung dispatcher
\include{content/hauptteil/umsetzungPoC/test/test}
%\section{Frontend}\label{sec:poc:frontend}
\begin{listing}[ht]
  \inputminted[linenos=true]{json}{content/hauptteil/umsetzungPoC/frontend/res/code/VueXstate.json}
  \caption{Datenstruktur Frontend}
  \label{list:dataStructFE}
\end{listing}
\begin{listing}[ht]
  \inputminted[linenos=true,breaklines=true]{json}{content/hauptteil/umsetzungPoC/frontend/res/code/VueXpageStruct.json}
  \caption{Datenstruktur Frontend - pageStruct}
  \label{list:dataStructPageStructFE}
\end{listing}
\begin{listing}[ht]
  \inputminted[linenos=true,breaklines=true]{json}{content/hauptteil/umsetzungPoC/frontend/res/code/VueXguiElements.json}
  \caption{Datenstruktur Frontend - guiElements}
  \label{list:dataStructGuiElements}
\end{listing}
\begin{listing}[ht]
  \inputminted[linenos=true,breaklines=true]{json}{content/hauptteil/umsetzungPoC/frontend/res/code/VueXdataNodes.json}
  \caption{Datenstruktur Frontend - dataNodes}
  \label{list:dataStructDataNodes}
\end{listing}
\begin{listing}[ht]
  \inputminted[linenos=true,breaklines=true]{json}{content/hauptteil/umsetzungPoC/frontend/res/code/VueXdataNodes.json}
  \caption{Datenstruktur Frontend - paramNodes}
  \label{list:dataStructParamNodes}
\end{listing}
sieht man in \refList{list:dataStructParamNodes}
