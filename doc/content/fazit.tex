\chapter{Zusammenfassung und Ausblick}\label{chapter:exit}
Die in der Zielsetzung formulierten wesentlichen Ziele sind leider nicht alle erreicht.
Die Architektur des \ac{scada} Systems ist entworfen und das \ac{poc} ist implementiert.
Dabei ist die Datenrate zwischen Frontend und Backend auf ein Minimum reduziert, 
da nach Aufbau der Seite nur noch die Änderungen der reinen Informationen ausgetauscht werden.
Es sind Steuerelemente implementiert, welche Darstellung sowie Änderung eines Prozesswerts ermöglichen.
Die Webapplikation (das Frontend) ist entsprechend ihres Datenmodells hinreichend modular, um das Instanzieren oder das Löschen von Steuerelementen zur Laufzeit zu ermöglichen.
Ein Neustart des Frontends wäre dazu nicht notwendig, allerdings ist diese Funktionalität wegen Zeitmangels im Backend nicht implementiert.
Das Zusammenspiel der implementierten Komponenten funktioniert sehr gut.
Beim Versuch die Echtzeitfähigkeit des Systems zu belegen, ist ein Problem mit der \ac{tcp} Verbindung aufgefallen.
Das Verhalten der Verbindung wurde analysiert und eine These entsprechend der Untersuchung aufgestellt.
Dieses Problem scheint lösbar zu sein, erfordert jedoch ein tiefergehendes Verständnis und eine explizite Untersuchung anhand der TCP/IP Protokolle sowie deren Verhalten. 
Dies kann in weiteren Arbeiten durchgeführt werden.