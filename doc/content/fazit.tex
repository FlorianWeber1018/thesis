\chapter{Zusammenfassung und Ausblick}\label{chapter:exit}
Die in der Zielsetzung formulierten wesentliche Ziele  sind leider nicht alle erreicht.
Die Architektur des \ac{scada} Systems ist entworfen und das \ac{poc} ist implementiert.
Dabei ist die Datenrate zwischen Frontend und Backend auf ein Minimum reduziert, 
da nach Aufbau der Seite nur noch die Änderungen der reinen Informationen ausgetauscht werden.
Es sind Steuerelemente implementiert, welche Darstellung sowie Änderung eines Prozesswerts ermöglichen.
Die Webapplikation (das Frontend) ist entsprechend ihreres Datenmodells hinreichend modular um das Instanzieren oder das Löschen von Steuerelementen zu Laufzeit zu ermöglichen.
Ein Neustart des Frontends wäre dazu nicht notwendig, allerdings ist diese Funktionalität wegen Zeitmangels im Backend nicht implementiert.
Das Zusammenspiel der implementierten Komponenten funktioniert sehr gut.
Beim Versuch die Echtzeitfähigkeit des Systems zu belegen, ist ein Problem mit der \ac{tcp} Verbindung aufgefallen.
Dieses Problem wurde analysiert und eine These entsprechend der Untersuchung aufgestellt.
Das Problem scheint lösbar und sei als Ausblick für weitere Arbeiten gegeben