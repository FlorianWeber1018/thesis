\usepackage[utf8]{inputenc}
\usepackage{a4}
\usepackage{csquotes}
\usepackage[numbers]{natbib}
\bibliographystyle{dcugerman}
\usepackage{pdfpages}

\usepackage{dsfont}
\usepackage{amsmath}
\usepackage{amsfonts}
\newcommand{\N}{\mathbb{N}}
\usepackage{graphicx}
\usepackage{here}
\usepackage[section]{placeins}
\usepackage[ngerman]{babel}
\usepackage[font=small,labelfont=bf,margin=\parindent]{caption}
\usepackage{subcaption}
\usepackage{setspace}

%Übersetzung des Literaturverzeichnisses
\newcommand{\literaturverz}[1]{
	%Autorenverknüpfung mit "und"
	\renewcommand{\harvardand}{und}
	\bibliography{#1}
}

\usepackage[colorlinks, pdfpagelabels, pdfstartview = FitH, bookmarksopen = true, bookmarksnumbered = true, linkcolor = black, plainpages = false, hypertexnames = false, citecolor = black, urlcolor = blue] {hyperref}
\usepackage{acronym}
%Paket für mehrspaltige Seiten
%--------------------------------------
\usepackage{multicol}
\setlength{\columnseprule}{0.2pt}
\setlength{\columnsep}{2em}
%--------------------------------------

%Paket für Syntaxhiglighting
%--------------------------------------
\usepackage{minted}
%\newmintedfile[inputsql]{sql}{
%    linenos,
%    autogobble,
%    breaklines,
%}

%\renewcommand\theFancyVerbLine{\scriptsize\arabic{FancyVerbLine}}
%--------------------------------------