\usepackage[utf8]{inputenc}
\usepackage{scrhack}
\usepackage[svgnames]{xcolor}
\usepackage{a4}
\usepackage[numbers]{natbib}
\bibliographystyle{dcugerman}

\usepackage{pdfpages}

\usepackage{dsfont}
\usepackage{amsmath}
\usepackage{amsfonts}
\newcommand{\N}{\mathbb{N}}
\usepackage{graphicx}
\usepackage{here}
\usepackage[section]{placeins}
\usepackage[ngerman]{babel}
\usepackage[font=small,labelfont=bf,margin=\parindent]{caption}
\usepackage{subcaption}
\usepackage{setspace}

%Übersetzung des Literaturverzeichnisses
\newcommand{\literaturverz}[1]{
	%Autorenverknüpfung mit "und"
	\renewcommand{\harvardand}{und}
	\bibliography{#1}
}
\definecolor{linkcolordef}{RGB}{1, 0, 0}
\usepackage[
  colorlinks=true, 
  linkcolor=black, 
  urlcolor=cyan, 
  citecolor=red, 
  pdfpagelabels, 
  pdfstartview = FitH, 
  bookmarksopen = true, 
  bookmarksnumbered = true,  
  plainpages = false, 
  hypertexnames = false]
  {hyperref}
\usepackage[printonlyused]{acronym}
%Paket für mehrspaltige Seiten
%--------------------------------------
\usepackage{multicol}
\setlength{\columnseprule}{0.2pt}
\setlength{\columnsep}{2em}
%--------------------------------------

%Paket für Syntaxhiglighting
%--------------------------------------
\usepackage{minted}
%\newmintedfile[inputsql]{sql}{
%    linenos,
%    autogobble,
%    breaklines,
%}




\renewcommand{\listingscaption}{Quellcodeauszug}
\renewcommand{\listoflistingscaption}{Quellcodeverzeichnis}
\newcommand{\refList}[1]{\listingscaption{} \ref{#1}}
\newcommand{\refFig}[1]{\figurename{} \ref{#1}}
\newcommand{\eigenName}[1]{\glqq \emph{#1}\grqq{}}

\newenvironment{code}{\captionsetup{type=listing}}{}

%\renewcommand\theFancyVerbLine{\scriptsize\arabic{FancyVerbLine}}
%--------------------------------------
\usepackage{csquotes}
